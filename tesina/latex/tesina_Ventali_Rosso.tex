\documentclass[a4paper, 14pt]{article}
\usepackage[T1]{fontenc}
\usepackage[utf8]{inputenc}
\usepackage[italian]{babel}
\usepackage{geometry}
\usepackage{imakeidx}
\usepackage{hyperref}
\usepackage{graphicx}
\pagestyle{myheadings}
\usepackage{titling}
\usepackage[fontsize=12pt]{fontsize}
\usepackage{chngcntr}
\usepackage{float}
\counterwithin{figure}{section}
\graphicspath{ {./img/} }

% Info documento
\date{A.A. 2023/2024}
\pagestyle{headings}
\geometry{a4paper, top=3cm, bottom=3cm, left=2.2cm, right=2.2cm}
\makeindex
\hypersetup{
	colorlinks,
	citecolor=black,
	filecolor=black,
	linkcolor=black,
	urlcolor=black
}


\title{\Huge {\textbf{UNITECNO}}}
\author{\Large Matteo Ventali, Stefano Rosso \\ \Large Ingegneria dell'Informazione (Sede di Latina) \\ \Large Università La Sapienza Roma }


% Inizio documento
\begin{document}
	% Intestazione
	\begin{figure}[t]
		\includegraphics[width=60mm]{logo.png}
	\end{figure}
	
	\maketitle	
	
	% Pagina che contiene l'indice del documento
	\newpage
	%\tableofcontents
	\tableofcontents
	
	% Inizio corpo documento
	\newpage
	\begin{flushleft}
		
	% Sezione 1
	\section{Introduzione e descrizione dei dati}
		\subsection{Introduzione}
			L'applicazione web UNITECNO ha come obiettivo la realizzazione di un negozio virtuale finalizzato alla vendita di dispositivi tecnologici, tramite un processo di fidelizzazione dei clienti.
			
			Il sito web offre un catalogo di prodotti, divisi per categoria,
			e documentati da immagini che ne riportano le specifiche tecniche, una breve descrizione e il prezzo. 
			La valuta all'interno del sistema è costituita dai \textit{crediti}.
			Il catalogo è fruibile da ogni utente del sito e può essere visionato secondo diversi criteri:
			\begin{itemize}
				\item prezzo;
				\item marca;
				\item modello.
			\end{itemize}
			
			\subsubsection{Categorie di dispositivi}
			Le categorie dei dispositivi sono:
			\begin{itemize}
				\item informatica;
				\item telefonia;
				\item TV e home;
				\item elettrodomestici;
				\item tempo libero;
			\end{itemize}
		
			\bigskip \textbf{Categoria informatica} \\ \smallskip
			Nella categoria informatica rientrano i seguenti dispositivi:
				\begin{itemize}
					\item computer portatili;
					\item computer fissi;
					\item tablet;
					\item stampanti;
					\item accessori.
				\end{itemize}
			
			\bigskip \textbf{Categoria telefonia} \\ \smallskip
			Nella categoria telefonia rientrano i seguenti dispositivi:
			\begin{itemize}
				\item smartphone e cellulari;
				\item telefoni fissi;
				\item accessori.
			\end{itemize}
				
			\bigskip \textbf{Categoria TV e home} \\ \smallskip
			Nella categoria TV e home rientrano i seguenti dispositivi:
			\begin{itemize}
				\item televisori;
				\item accessori.
			\end{itemize}
			
			\bigskip \textbf{Categoria elettrodomestici} \\ \smallskip
			Nella categoria elettrodomestici rientrano i seguenti dispositivi:
			\begin{itemize}
				\item lavatrici;
				\item frigoriferi;
				\item lavastoviglie;
				\item piani cottura;
				\item forni;
				\item condizionatori;
				\item accessori.
			\end{itemize}			
			
			\bigskip \textbf{Categoria tempo libero} \\ \smallskip
			Nella categoria tempo libero rientrano i seguenti dispositivi:
			\begin{itemize}
				\item cuffie e auricolari;
				\item fotocamere e videocamere;
				\item console da gioco;
				\item accessori.
			\end{itemize}			
			
			
			
			\subsubsection{Categorie di utenti}
			Gli utenti si dividono in:
			\begin{itemize}
				\item non registrati, da qui in avanti denominati \textit{visitatori};
				\item registrati, da qui in avanti denominati \textit{clienti};
				\item \textit{gestori};
				\item amministratori, da qui in avanti denominati \textit{admin}.
			\end{itemize}
			
			\bigskip \textbf{Visitatori} \\ \smallskip
			L'utente visitatore può:
			\begin{itemize}
				\item accedere al catalogo;
				\item accedere alla sezione contatti;
				\item visionare le FAQ;
				\item registrarsi.
			\end{itemize}
			
			\bigskip \textbf{Clienti} \\ \smallskip
			Ogni cliente usufruisce di due portafogli:
			\begin{itemize}
				\item \underline{portafoglio standard}, ricaricabile;
				\item \underline{portafoglio bonus}, alimentato a seguito di acquisti effettuati.
			\end{itemize}
			
			L'utente cliente può:
			\begin{itemize}
				\item fare tutto quello che è permesso al visitatore, ma non può registrarsi;
				\item accedere al profilo, visionare gli acquisti effettuati, quelli in corso, il portafoglio
				standard e quello bonus, la propria reputazione e può vedere e gestire i propri dati personali;
				\item acquistare dal catalogo, inserendo i prodotti nel carrello;
				\item scrivere recensioni, domande e risposte;
				\item giudicare gli interventi degli altri clienti;
				\item ricaricare il portafoglio standard attraverso una richiesta all'admin.
				L’indirizzo di consegna è di default quello indicato dall’utente nel profilo, ma può essere
				modificato prima di finalizzare l’acquisto.
			\end{itemize}
		
			\bigskip \textbf{Gestori} \\ \smallskip
			L'utente gestore può:
			\begin{itemize}
				\item aggiungere, eliminare, modificare prodotti in catalogo;
				\item definire gli sconti sui prodotti;
				\item consultare il profilo dei clienti;
				\item moderare e dare un giudizio ai contributi dei clienti;
				\item rispondere alle domande;
				\item aggiungere domande e risposte alle FAQ;
				\item elevare una domanda dei clienti alle FAQ, con la risposta migliore.
			\end{itemize}
		
			\bigskip \textbf{Admin} \\ \smallskip
			L'utente admin può:
			\begin{itemize}
				\item visionare e modificare i dati anagrafici e di accesso degli utenti;
				\item bannare e attivare i clienti;
				\item accettare o rifiutare le richieste di ricarica dei clienti;
				\item aggiungere domande e risposte alle FAQ;
				\item elevare una domanda dei clienti alle FAQ, con la risposta migliore.
			\end{itemize}
		
		\subsection{Note}
			\subsubsection{Portafogli}
				Nell'applicazione, associati ad ogni cliente, verranno gestiti due portafogli:
				\begin{itemize}
					\item \underline{portafoglio standard};
					\item \underline{portafoglio bonus}.
				\end{itemize}

				\bigskip \textbf{Portafoglio standard} \\ \smallskip			
				Il portafoglio standard viene ricaricato attraverso l’acquisto dei crediti mediante richiesta tramite form.
				L’utente può scegliere tra dei tagli ben definiti: nella form sarà presente un menù a tendina in cui poter selezionare il l'entità del versamento da effettuare.
				
				\bigskip \textbf{Portafoglio bonus} \\ \smallskip
				Il portafoglio bonus si alimenta in due diversi modi:
				\begin{itemize}
				\item ottenendo una quota di crediti da ciascun acquisto effettuato (\textit{bonus fisso});
				\item ottenendo crediti bonus per ogni acquisto relativo ad un prodotto in offerta speciale (\textit{bonus speciale}).
				\end{itemize}
			
			\subsubsection{Sconti e offerte}
				Uno sconto è una percentuale di riduzione del prezzo e può essere di tre tipi:
				\begin{itemize}
					\item \underline{offerta speciale}, per uno specifico prodotto in un determinato periodo temporale;
					\item \underline{fisso}, calcolato sulla base delle transazioni effettuate dall’utente;
					\item \underline{variabile}, in base alle scelte effettuate dal cliente sull’utilizzo dei crediti bonus (disponibili nel suo portafoglio bonus).
				\end{itemize}
			
				
				\bigskip \textbf{Offerta speciale} \\ \smallskip
				L'offerta speciale viene creata da un gestore che ne definisce il prodotto target e la sua durata temporale.
				Lo sconto in oggetto, se presente, blocca la possibilità al cliente di godere dello sconto fisso e variabile.
				
				\bigskip \textbf{Sconto fisso} \\ \smallskip
				Lo sconto fisso si basa sull'analisi degli acquisti effettuati dal cliente. \\
				In fase di analisi si tiene conto dei seguenti parametri:
				\begin{itemize}
					\item ammontare N dei crediti spesi dal cliente;
					\item ammontare M dei crediti spesi dal cliente nell'anno solare corrente;
					\item reputazione del cliente;
					\item periodo decorso dall'iscrizione del cliente.
				\end{itemize}
				La funzione per calcolare lo sconto fisso è:
				
				
	\end{flushleft}
\end{document}